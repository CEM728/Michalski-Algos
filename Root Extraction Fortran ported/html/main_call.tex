
% This LaTeX was auto-generated from MATLAB code.
% To make changes, update the MATLAB code and republish this document.

\documentclass{article}
\usepackage{graphicx}
\usepackage{color}

\sloppy
\definecolor{lightgray}{gray}{0.5}
\setlength{\parindent}{0pt}

\begin{document}

    
    
\subsection*{Contents}

\begin{itemize}
\setlength{\itemsep}{-1ex}
   \item PEC Example
   \item Open Dielectric
   \item Test Case 1
   \item Gold-film example
   \item Wilkinson Polynomial
   \item TLGF
   \item Code paramters
   \item ROOT COUNTING and CONTOUR INTEGRALS
   \item BUILD HANKEL MATRICES
   \item SOLVE EIGENVALUE PROBLEM
   \item USE NEWTON / HALLEY'S METHOD TO REFINE THE ROOTS
\end{itemize}
\begin{verbatim}
clear; clc;tic
%  ---- Here we declare the namelist /INPUT/ that defines the search parameters.
%
%       It is useful to remember that
%
%
%          point(1)  defines the REAL value of the lower left hand co-ordinate
%                    of the rectangle
%
%          point(2)  defines the IMAGINARY value of the lower left hand co-ordinate
%                    of the rectangle
%
%          step(1)   defines the length of the rectangle on the REAL axis.
%
%          step(2)   defines the length of the  rectangle on the IMAGINARY axis
%
%          droot_converged  We declare that a root has been found whenever
%                           ABS(f(z)) < droot_converged
%
%          npts      Number of points in per side of rectangle in the fixed
%                    integration quadrature
%
%
\end{verbatim}


\subsection*{PEC Example}

\begin{par}
f = 1e12; omega = 2*pi*f; lambda = 3e8/f; \% Material Properties ep1 = 12; ep2 = 4 ; ep3 = 2.1; ep4 = 1;
\end{par} \vspace{1em}
\begin{par}
\% EM constants mu0 = 4*pi*1e-7; ep0 = 8.854e-12;
\end{par} \vspace{1em}
\begin{par}
\% Propagations Constants k1 = omega*sqrt(mu0*ep0*ep1); k2 = omega*sqrt(mu0*ep0*ep2); k3 = omega*sqrt(mu0*ep0*ep3); k4 = omega*sqrt(mu0*ep0*ep4);
\end{par} \vspace{1em}


\subsection*{Open Dielectric}

\begin{verbatim}
f = 1e12;
omega = 2*pi*f;
lambda = 3e8/f;
% Material Properties
ep1 = 1; % Air
ep2 = 9.7 ; % GaN/AlGaN layers combined
ep3 = 11; % Silicon base


% EM constants
mu0 = 4*pi*1e-7;
ep0 = 8.854e-12;

% Propagations Constants
k1 = omega*sqrt(mu0*ep0*ep1);
k2 = omega*sqrt(mu0*ep0*ep2);
k3 = omega*sqrt(mu0*ep0*ep3);

% point = -.20 -.20i;
% step = .40 + .40i;

% % For Dellnitz function with A B C T
% point = .95*k_air -.05*k_air*1i;
% % point = 0;
% step = .25*k_air + .12*k_air*1i;
\end{verbatim}


\subsection*{Test Case 1}

\begin{par}
point = -2.2 - 1i*3.5; step = 2.5 + 1i*8;
\end{par} \vspace{1em}


\subsection*{Gold-film example}

\begin{par}
point = 1.9 - 1i*1.5; step = 0.4 + 1i*.20;
\end{par} \vspace{1em}


\subsection*{Wilkinson Polynomial}

\begin{par}
point = 5.5 - 1i*.5; step = 1.0 + 1i*1;
\end{par} \vspace{1em}


\subsection*{TLGF}

\begin{verbatim}
point = 1.2*k1 - 1.2i*k1;
step = 1.04*k1 + 2.04i*k1;

% point = -20.3 -20.7i;
% step = 40.6 + 41.4i;

% Import from Namelist
% point  = -2.2 - 3.5*1i;
% step   = 5.0 + 8.0 *1i;

% point = -5 -5i;
% step = 10 + 10i;

%----------------------------------------------------------------------
\end{verbatim}


\subsection*{Code paramters}

\begin{verbatim}
%----------------------------------------------------------------------

% Convergence Criterion
droot_converged = 1e-14;

% Number of points on each side of the contour
% Increase it to ensure capturing all Riemann sheets
npts = 32768*2;

% Currently not used as splitting algorithm not implemented
maxboxes = 500;

% Limit the number of roots per box
% If this is greater, split
max_roots_per_box = 5;

% Maxium roots to be found by the routine
maxroots = 500;



%----------------------------------------------------------------------
\end{verbatim}


\subsection*{ROOT COUNTING and CONTOUR INTEGRALS}

\begin{verbatim}
%----------------------------------------------------------------------
% Count the number of roots within the rectangle now by examination
% of the change in the argument and also compute the integrals "s",
% defined in the paper, need to construct the matrix eigenvalue
% problem.
%
% The contour integral is computed through MATLAB's integral function
% using way-points to define the path of integration

[nroots, s] = countz (point, step, npts, maxroots);


%----------------------------------------------------------------------
\end{verbatim}


\subsection*{BUILD HANKEL MATRICES}

\begin{verbatim}
%----------------------------------------------------------------------
%
% Construct H and H1 from the computed contour integrals s_n
%

for k = 1 : nroots
    for l = 1 : nroots
        H1(k,l) = s(k + l);
        H(k,l) = s(k + l - 1 );
    end
end

%----------------------------------------------------------------------
\end{verbatim}


\subsection*{SOLVE EIGENVALUE PROBLEM}

\begin{verbatim}
%----------------------------------------------------------------------
% The eigenvalue problem looks like
%
% H - \lambda \time H1 = 0
%
% The eigenvalues, \lambda are the initial roots of the system
%
zinitial_roots = eig(H1,H,'qz');
%
% Check the quality of initial roots
%
zinitial_func = FZ(zinitial_roots);
%
\end{verbatim}


\subsection*{USE NEWTON / HALLEY'S METHOD TO REFINE THE ROOTS}

\begin{verbatim}
qpl = point;
%
qpt = point + step;
%
% Check if we have a converged solution at this stage
% for each root. If not, apply Newton's / Halley's method
%
for k = 1 : nroots
    %
    % If initially obtained roots are good enough
    %
    if(abs(zinitial_func(k)) < droot_converged)

        zfinal_roots(k) = zinitial_roots(k);
        zfinal_func(k)  = zinitial_func(k);
        %
        % Otherwise call Halley's method
        %
    else

        zfinal_roots(k) = newtzero(@FZ, zinitial_roots(k));
        %
        % Compute the value of the function at the converged root.
        %
        zfinal_func(k) = FZ(zfinal_roots(k));
    end
end
zfinal_func = zfinal_func';
zfinal_roots = zfinal_roots';
toc
\end{verbatim}

        \color{lightgray} \begin{verbatim}Elapsed time is 0.313800 seconds.
\end{verbatim} \color{black}
    


\end{document}
    
